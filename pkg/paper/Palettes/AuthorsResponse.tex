\documentclass[a4paper]{article}
\usepackage{a4wide}

\begin{document}

\textbf{Reviewer 1}

\begin{itemize}
  \item \textit{It would be helpful to point out that the CIE Luv colour
        space is only roughly perceptually uniform -- it is a compromise
	between perceptual uniformity and simplicity of conversion formulas.
	According to the authors' reference Poynton (2000) it has a non-uniformity
	of 6:1 compared to 80:1 for the XYZ colour space -- I don't know what the
	non-uniformity is for HSV, but gamma correction presumably means that it
	is smaller than for XYZ.}

        Paul?
	
  \item \textit{The article really should have better references to the scientific
        literature on colour. Lumley (2006) and Ihaka (2003) are useful, especially
        the latter, but neither is really peer-reviewed literature.  I don't know
        what the journal's policy is on Wikipedia as a reference [although this is
        the sort of topic where Wikipedia typically does quite well]}
	
	We added some references (Paul, Kurt?) but also kept the Wikipedia
	references because they are quite helpful as overviews.
\end{itemize}


\textbf{Reviewer 2}

\begin{itemize}
  \item \textit{p.7 line 9fb. There is confusion here. ``Warm colors
        (from the blue/green part of the spectrum...) and cold colors
	(from the yellow/red ....)''}
	
	fixed
	
  \item \textit{p.13 Harezlak reference. In all other articles, key words in the article
        title are capitalized, except in this reference. I personally would prefer
        if only the lead word in the article title were capitalized.}
	
	fixed (capitalized in the underlying {\sc Bib}{\TeX} file, the rest
	will be handled by Elsevier's style files)
	
  \item \textit{p.14 Murrell reference. ``Chapmann" should be ``Chapman". I guess a little
        germaninzation is not too bad :-)}
	
	Maybe Tschappmann? :-) More seriously: fixed, and thanks for spotting this.
\end{itemize}


\textbf{3rd party review}

\begin{itemize}
  \item \textit{This is a potentially useful paper, although right now it needs more
        work.  The paper is lacking several common statistics examples: a
        scatterplot, barchart or mosaic plot, to introduce and establish the
        credentials of the color choices. Cindy Brewer's work covers maps very
        nicely. But the big problem in using these methods for statistical
        graphics is that we don't have large swatches of adjoining colour.}
	
	New section: Illustrations.
	
  \item \textit{Its clear that the HCL is the better approach, as described in Ihaka's
        work. I have a difficult time determining how this paper expands on
        Ihaka's work to make substantial new contributions.}
	
	Ihaka's work just covered qualitative palettes (as referenced in Section~4.1),
	the other color selection strategies are novel.

  \item \textit{The defaults in statistical graphics packages should use of these
        ideas to create better initial color choices for the user. One would
        expect that the software that these authors have provided will make it
        easier for new developers to build good color defaults into their
        plots. But its not clear that it would be easy to use the vcd
        functions, or that all new packages should depend on vcd. Wouldn't it
        be more accessible to R users if there was a separate color package,
        or part of colorspace, that would make these color scales available?}
	
	This is a good point! All functions have been moved from vcd to colorspace.

  \item \textit{The color defaults in ggplot2 (http://had.co.nz/ggplot2/) are based on
        some similar ideas as described in this paper. I'm not sure what the
        overlap or connection is, but ggplot2 provides an excellent example of
        great default color schemes, for all sorts of statistical graphics.}
	
	I think Hadley copied some code (in order not to depend on vcd), following
	my suggestion to do so. We could also cite ggplot2.

\end{itemize}

\end{document}
