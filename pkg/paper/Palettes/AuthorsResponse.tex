\documentclass{wuletter}
\signature{Achim Zeileis, Kurt Hornik, Paul Murrell}
\usepackage[american]{babel}
\renewcommand{\rmdefault}{phv}

\begin{document}
\begin{letter}{
Erricos~J.~Kontoghiorghes\\
School of Computer Science and Information Systems\\
Birkbeck College\\
University of London\\
Malet Street, London WC1E 7HX, United Kingdom}

\concerning{Revision of CSDA-D-07-00860}

\opening{Dear Prof.~Kontoghiorghes,}

please find attached the revision of our manuscript
\begin{quote}
\textit{Escaping RGBland: Selecting Colors for Statistical Graphics}
\end{quote}
for \emph{Computational Statistics \& Data Analysis}. 

We would like to thank you, the associate editor and the referees
for the most helpful reviews of our work. We tried to incorporate
all of the suggestions. In particular, we have complemented the
examples from Section~2 (heatmaps, maps) by a set of further illustrations
in a new Section~5 (including pie charts, highlighted mosaic displays,
scatter plots, and mosaic displays with residual-based shading).
Furthermore, we have included some more references to the color and
computer graphics literature. A detailed point-to-point reply is
included below.

\closing{Best regards,}
\end{letter}

\newpage

\textbf{Reviewer 1}

\begin{itemize}
  \item \textit{It would be helpful to point out that the CIE Luv colour
        space is only roughly perceptually uniform -- it is a compromise
	between perceptual uniformity and simplicity of conversion formulas.
	According to the authors' reference Poynton (2000) it has a non-uniformity
	of 6:1 compared to 80:1 for the XYZ colour space -- I don't know what the
	non-uniformity is for HSV, but gamma correction presumably means that it
	is smaller than for XYZ.}\\[-0.3cm]

        Yes, uniformity is a bit more complicated. Hence, we refrain from 
	discussing its details explicitly here, and have instead included several
	remarks in Sections~1 and 3 conveying some limitations of the CIELUV model.
	
  \item \textit{The article really should have better references to the scientific
        literature on colour. Lumley (2006) and Ihaka (2003) are useful, especially
        the latter, but neither is really peer-reviewed literature.  I don't know
        what the journal's policy is on Wikipedia as a reference [although this is
        the sort of topic where Wikipedia typically does quite well]}\\[-0.3cm]
	
	We have included additional pointers to Kaiser and Boynton (1996) and Knoblauch (2002)
	at the beginning of Section~3, and also included some further references
	to the computer graphics literature in other sections. We have kept the
	Wikipedia references because we agree with the referee that they are
	helpful (especially as a first overview).
\end{itemize}


\textbf{Reviewer 2}

\begin{itemize}
  \item \textit{p.7 line 9fb. There is confusion here. ``Warm colors
        (from the blue/green part of the spectrum...) and cold colors
	(from the yellow/red ....)''}\\[-0.3cm]
	
	fixed
	
  \item \textit{p.13 Harezlak reference. In all other articles, key words in the article
        title are capitalized, except in this reference. I personally would prefer
        if only the lead word in the article title were capitalized.}\\[-0.3cm]
	
	fixed
	
  \item \textit{p.14 Murrell reference. ``Chapmann" should be ``Chapman". I guess a little
        germaninzation is not too bad :-)}\\[-0.3cm]
	
	Maybe Tschappmann? :-) More seriously: fixed, and thanks for spotting this.
\end{itemize}


\textbf{3rd party review}

\begin{itemize}
  \item \textit{This is a potentially useful paper, although right now it needs more
        work.  The paper is lacking several common statistics examples: a
        scatterplot, barchart or mosaic plot, to introduce and establish the
        credentials of the color choices.}\\[-0.3cm]
	
	We have added a new Section~5 ``Illustrations'' that complements
	the displays from Section~2 (heatmaps, maps) by presenting further
	applications of the suggested color palettes (pie charts,
	highlighted mosaic displays, scatter plots, and mosaic displays
	with residual-based shading).
	
  \item \textit{Its clear that the HCL is the better approach, as described in Ihaka's
        work. I have a difficult time determining how this paper expands on
        Ihaka's work to make substantial new contributions.}\\[-0.3cm]
	
	Ihaka's work just covered qualitative palettes (as referenced in Section~4.1),
	the other color selection strategies are novel.

  \item \textit{The defaults in statistical graphics packages should use of these
        ideas to create better initial color choices for the user. One would
        expect that the software that these authors have provided will make it
        easier for new developers to build good color defaults into their
        plots. But its not clear that it would be easy to use the vcd
        functions, or that all new packages should depend on vcd. Wouldn't it
        be more accessible to R users if there was a separate color package,
        or part of colorspace, that would make these color scales available?}\\[-0.3cm]
	
	This is a good idea! All functions have been moved from vcd to colorspace,
	as of vcd~1.2-0 and colorspace~0.97.

  \item \textit{The color defaults in ggplot2 (http://had.co.nz/ggplot2/) are based on
        some similar ideas as described in this paper. I'm not sure what the
        overlap or connection is, but ggplot2 provides an excellent example of
        great default color schemes, for all sorts of statistical graphics.}\\[-0.3cm]
	
	ggplot2 also employs HCL color palettes by default, based on similar ideas.
        ggplot2 has its own code for this---my understanding is that this is 
	in part to avoid depending on vcd. We have included a pointer to ggplot2
	in Section~6.

\end{itemize}

\end{document}
